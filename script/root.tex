\documentclass[journal,twoside,web]{ieeecolor}
% \documentclass[12pt,peerreview,draftversion,onecolumn,print]{ieeecolor}
\input{preamble.tex}
\usepackage{generic}

% Not sure if needed
\def\BibTeX{{\rm B\kern-.05em{\sc i\kern-.025em b}\kern-.08em
    T\kern-.1667em\lower.7ex\hbox{E}\kern-.125emX}}

\markboth{\journalname, VOL. XX, NO. XX, May 2025}
{Satici: John Chiasson Memorial Talk Script
(May 2025)}

\begin{document}

\title{Script: John Nelson Chiasson} 
\author{
    Aykut C. Satici \IEEEmembership{Member, IEEE}
%    \thanks{Not submitted for review in May 2025.}
    \thanks{A. C. Satici is with the Mechanical and Biomedical Engineering Department, Boise State University, Boise, ID 83706 USA
    (e-mail: aykutsatici@boisestate.edu).}
}
\maketitle
% \IEEEpeerreviewmaketitle

\section{Slide 1}
%
My short presentation cannot do justice to this great man, John Nelson Chiasson, who passed away on February 26. 2025.

The title is taken from a Facebook page that I stumbled upon approximately $4$ 
years ago - John Chiasson: An American Treasure. It was such a great description
that I printed out a few pictures from this FB page and hung them 
on the desk he used to work at with his mentee, Mansi, who is also with us
today, having travelled from California.

\section{Slide 2}

I got to know John when I interviewed here at Boise State back in Fall 2016. 
During my job talk, John noticed a mistake in one of my slides and pointed it out to me.

Much later, I got to meet some of the members of his family; Laura and Tim, in 
Istanbul. I met his twin brother Bill, about whom I had heard so much, when John
sadly passed away. Bill and Laura sent me a bunch of pictures from their stack, 
which I really appreciated and sprinkled throughout this presentation. This is 
the phase of John's life I had no idea about. Look how fit and handsome he was 
in his youth!

Bill mentioned how important this wand was to both him and John. They found this
wand during the time they spent in Africa while their father was taking a
sabbatical there.

Even though John and I met 8-9 years ago, we shared a lot of fond memories. This
picture on the left is taken at Mountain Home, where there is a US air base. We 
have been working on several projects in collaboration with Pitch Aeronautics, 
whose CEO, Zach Adams, is also an F-15 pilot! He had invited us for his last 
flight before he took a sabbatical year from the Air Force.

The picture on the right is from last year, when John invited my wife and me to 
his house and cooked us a dinner, whose recipe he learned from his Swiss friend,
Thomas Keller.

This picture on the left is taken in Stanley, ID, when we visited in 2021, with 
our research team. We were there just this last Saturday to disperse John's 
ashes at one of his favorite places to visit.

\section{Slide 3}

John was, among other things, a brilliant academician. The seeds of this brilliance were planted when he attended the University of Arizona to study 
mathematics. He used to tell me that he always liked math because everything was
well-defined and there was no room for hand-wavy arguments - one of John's most
important traits:

\begin{quote}
John was a rigorous man - he did not tolerate artificial, phony arguments or
opinions.
\end{quote}

He then switched gears to become a master engineer at Washington State
University, and a doctor of Control Sciences at Unviersity of Minnesote Twin
Cities.

\section{Slide 4}

John went on to produce a brilliant academic career. He was a top $1\%$ scientist according to \ldots had an h-index of $52$!

I had the honor and pleasure to produce his final journal paper, working with
him on the topic of state estimation and drone navigation around power lines
using the magnetic field produced by the three-phase currents. This was a
project we completed in collaboration with Pitch Aeronautics.


\section{Slide 5}

John started work with Zach Adams, who had recently founded Pitch Aeronautics, 
a company that utilizes a novel drone with a cyclorotor to install an uninstall 
sensors and other equipment on power lines. Zach, who is a very prolific and 
technically capable person is also an F-15 pilot. John worked with Mansi and
Zach to develop the basic flight controller for the Astria drone. I have joined
the team later on to work on other important aspects of the project and the 
company has grown so much since then.

John has always been a great mentor, not just to his students, but also my 
mentees. Here you see him hooding Wankun, my first PhD student, with me. You 
have just heard from his first PhD student Ruthvik, who is now leading a 
machine learning team at Lamb Weston.


\section{Slide 6}

If John had warmed up to you enough, he would show you a hidden side of
him - his incredible sense of humor. Here, you see a couple of the comics he
shared with me over email or phone correspondences.

\section{Slide 7}

He valued his friends and family very much. He would take a lot of time coming 
up with witty jokes to celebrate their birthdays as you can see in these videos.

\section{Slide 8}

There is so many more sides to John that I could not cover in this short time.
But for today, I will end with one last but not the least important side of John
- his love for travel and appreciation of nature.

John came to Sabanci University in Istanbul, Turkey, for a semester-long 
sabbatical in Fall 2024. He enjoyed the cafeteria food very much. He and I went 
to many museums, both in Istanbul and Ankara. Here you see us in Kadikoy and 
at Ataturk's mausoleum in Ankara. 

Here in Boise, we took John several times to rafting trips on the Payette river.
I remember pulling him out of the water once when he fell off \ldots

Here again,  you see us in Turkey. The left photo is taken at the Dolmabahce Palace in Istanbul. This is where the last Ottoman Sultans lived. In the 
right photo, we are seen in the high-speed train that is taking us from Istanbul to Ankara.


\section{Slide 9}

Lastly, here you see the pictures I have taken of John's office on the day 
he passed away and just a week ago. Those geometric figures on the left were, 
for the most part, drawn by me. On the right, you can see how many students 
he touched during just the last few years.

John was my friend, my mentor, my colleague, my role model, and a second father 
to me over the last 8 years. I am and will be missing him every day of the 
rest of my life. A part of him will live on in my heart until the day I die.

Goodbye, Old Man! We loved you so, so much!

% \begin{abstract} % Abstract of not more than 200 words.
    This technical report demonstrates the implementation of a decision 
    transformer (DT) for learning the shortest paths on graphs.
\end{abstract}

% \begin{IEEEkeywords}
%     decision transfomer, machine learning, generative pretraining
% \end{IEEEkeywords}

% makeindex root.nlo -s nomencl.ist -o root.nls
% \nomenclature[aa]{$a, b$}{Indices referring to arbitrary frames (e.g. 
$\Sigma_A$, $\Sigma_B$)}
\nomenclature[ab]{$\Sigma_0$}{Inertial coordinate frame with origin $o_0$}
\nomenclature[ab]{$\Sigma_P$}{Pole coordinate frame with origin $o_p$}
\nomenclature[ac]{$\Sigma_B$}{Drone coordinate frame at the CoM $o_b$}
\nomenclature[ad]{$\Sigma_C$}{Contact frame with origin $o_c$}
\nomenclature[ae]{$\bm{p}_{ab}$}{Pos. vector from $o_a$ to $o_b$}
\nomenclature[af]{$\bm{R}_{ab}$}{Rotation matrix transforming ($\Sigma_B
\rightarrow \Sigma_A$)}
\nomenclature[ag]{$\bm{T}_{ab}$}{Hom. transformation ($\Sigma_B \rightarrow 
\Sigma_A$)}
\nomenclature[ah]{$\bm{v}_{ab}$}{Body linear velocity of $\Sigma_B$ 
($=\bm{R}_{ab}^\top \dot{\bm{p}}_{ab}$)}
\nomenclature[ai]{$\bm{\omega}_{ab}$}{Body angular velocity of $\Sigma_B$ w.r.t. 
$\Sigma_A$}
\nomenclature[aj]{$\bm{V}_{ab}$}{Body twist of $\Sigma_B$ w.r.t. 
$\Sigma_A$($= \bmat{\bm{\omega}^\top_{ab} & \bm{v}^\top_{ab}}^\top$)} 
\nomenclature[ba]{$\bm{f}$}{Drone thrust force ($ = f \bm{z}_b$, $f < 0$, 
specified in $\Sigma_B$)}
\nomenclature[bb]{$\bm{\tau}$}{Drone torque (specified in $\Sigma_B$)}
\nomenclature[bc]{$\bm{\mc{F}}$}{Drone wrench ($= \bmat{ 
\bm{\tau}^\top & \bm{f}^\top}^\top$)}
\nomenclature[ca]{$\bm{f}_e$}{Contact force on the drone (specified in 
$\Sigma_C$)}
\nomenclature[cb]{$\bm{\tau}_e$}{Contact torque on the drone (specified in 
$\Sigma_C$)}
\nomenclature[cc]{$\bm{\mc{F}}_e$}{Contact wrench on the drone (specified in 
$\Sigma_C$)}
\nomenclature[za]{$\bm{g}$}{Gravitational acceleration vector (specified in
$\Sigma_0$)}
\nomenclature[zb]{$m$}{Mass of the drone}
\nomenclature[zc]{$\bm{J}$}{Inertia matrix of the drone (specified in
$\Sigma_B$)}
\nomenclature[zd]{$m_w$}{Mass of the wire}
\nomenclature[ze]{$J_w$}{Inertia matrix of the wire about the pole along the
$\bm{y}_p-$axis}
\printnomenclature


\end{document}
