\documentclass[journal,twoside,web]{ieeecolor}
% \documentclass[12pt,peerreview,draftversion,onecolumn,print]{ieeecolor}
\usepackage[usenames,table,x11names,svgnames,dvipsnames]{xcolor}
\usepackage[export]{adjustbox}
\usepackage{algorithm}
\usepackage[noend]{algpseudocode}
\usepackage{amsmath,amssymb,amsfonts}
\usepackage[USenglish]{babel}
\usepackage{bigints}
\usepackage{bm}
\usepackage{booktabs}
\usepackage{cancel}
\usepackage[tableposition=above]{caption}
% \usepackage{centernot}
% \usepackage{comment}
% \usepackage{enumitem}
\usepackage{epsfig}
\usepackage{epstopdf}
% \usepackage[letterpaper, top=1.0in, bottom=1.0in, left=1.0in, right=1.0in]{geometry}
\RequirePackage[OT1]{fontenc}
% \usepackage{fontspec}
\usepackage{graphics}
\usepackage{graphicx}
\graphicspath{{figures/}}
% \usepackage{ifpdf}
% \usepackage{lastpage}
% \usepackage{leftidx}
\usepackage{lipsum}
% \usepackage{mathrsfs}
\usepackage{mathtools}
% \usepackage{multicol}
% \usepackage{multirow}
\usepackage{nicefrac}
% \usepackage{nicematrix}
\usepackage{nomencl}
\makenomenclature
% \usepackage{pgfplots}
\usepackage{pifont}
% \usepackage{ragged2e}
% \usepackage{rotating}
% \usepackage{stmaryrd}
\usepackage{siunitx}
\usepackage{soul}
\usepackage[caption=false]{subfig}
% \usepackage[]{svg}
\usepackage{tabularx}
\usepackage[]{threeparttable}
\usepackage{tikz}
% \usepackage{tkz-euclide}
% \usepackage{ctable}
% \usetikzlibrary{matrix, arrows}
\usetikzlibrary{shapes.geometric, arrows, decorations.markings, shapes.arrows}
\usepackage[textsize=footnotesize]{todonotes}
% \usepackage{wrapfig}

\tikzstyle{startstop} = [rectangle, rounded corners, minimum width=1cm, minimum
height = 0.5cm, text centered, draw=black, fill=red!30]
\tikzstyle{io} = [trapezium, trapezium left angle=70, trapezium right angle=110,
minimum height=1cm, text width=3cm, text centered, draw=black, fill=blue!30]
\tikzstyle{process} = [rectangle, minimum width=2cm, minimum height=0.8cm, text
centered, text width=2cm, draw=black, fill=orange!30]
\tikzstyle{decision} = [diamond, aspect=1.25, minimum width=2cm, minimum height=0.5cm, 
text centered, text width=3cm, draw=black, fill=green!30]
\tikzstyle{arrow} = [thick, ->, >=stealth]

\newcommand\encircle[1]{%
  \tikz[baseline=(X.base)] 
    \node (X) [draw, shape=circle, inner sep=0] {\strut #1};}


\makeatletter
\newcommand{\rmnum}[1]{\romannumeral #1}
\newcommand{\Rmnum}[1]{\expandafter\@slowromancap\romannumeral #1@}
\makeatother

\newcommand{\bmat}[1]{\begin{bmatrix}#1\end{bmatrix}}
\newcommand{\pmat}[1]{\begin{pmatrix}#1\end{pmatrix}}
\newcommand{\ubar}[1]{\text{\b{$#1$}}}
\newcommand{\norm}[2]{\|{#1}\|_{{}_{#2}}}
\newcommand{\abs}[1]{\left|{#1}\right|}
\newcommand{\mbf}[1]{\mathbf{#1}}
\newcommand{\mc}[1]{\mathcal{#1}}
\newcommand{\dd}{\operatorname{d}\!}
\newcommand{\lininterp}{\operatorname{LinInterp}\!}
\newcommand{\muc}[2]{\multicolumn{#1}{c}{#2}}
\newcommand*\Eval[3]{\left.#1\right\rvert_{#2}^{#3}}
\newcommand{\inner}[1]{\left\langle#1\right\rangle}
\newcommand{\pd}[2]{\frac{\partial #1}{\partial #2}}
\newcommand{\pdd}[2]{\frac{\partial^2 #1}{\partial #2^2}}
\newcommand{\el}[2]{\frac{\dd}{\dd t}\pd{\mc{L}}{\dot{#1}} - \pd{\mc{L}}{#1} = #2}
\newcommand{\elk}[2]{\frac{\dd}{\dd t}\pd{\mc{L}}{\dot{#1}_k} - \pd{\mc{L}}{#1_k} = #2_k}
\newcommand{\vectornorm}[1]{\left|\left|#1\right|\right|}
\newcommand{\dom}[1]{\textrm{dom}\;#1}
\newcommand{\bx}{{\bf x}}
\newcommand{\bu}{{\bf u}}
\newcommand{\cmark}{\ding{51}}%
\newcommand{\xmark}{\ding{55}}%
\newcommand*{\vertbar}{\rule[-1ex]{0.5pt}{2.5ex}}
\newcommand*{\horzbar}{\rule[.5ex]{2.5ex}{0.5pt}}

\newcommand{\idapbc}{\textsc{IdaPbc}}
\newcommand{\electric}{{\textcolor{blue}{\hspace{-0.5mm}$\bm{E}$\;}}}
\newcommand{\magnetic}{{\textcolor{red}{\hspace{-0.5mm}$\bm{B}$\;}}}

\newcommand{\rotvert}{\rotatebox[origin=c]{90}{$\vert$}}
\newcommand{\rowsvdots}{\multicolumn{1}{@{}c@{}}{\vdots}}
\newcommand{\brows}[1]{%
  \begin{bmatrix}
  \begin{array}{@{\protect\rotvert\;}c@{\;\protect\rotvert}}
  #1
  \end{array}
  \end{bmatrix}
}

% \theoremstyle{plain}
% \newtheorem{thm}{Theorem}[section]
% \makeatletter
% \@addtoreset{thm}{section}
% \makeatother
% \newtheorem{cor}[thm]{Corollary}
\newtheorem{lem}{Lemma}
% \newtheorem{claim}[thm]{Claim}
% \newtheorem{axiom}[thm]{Axiom}
% \newtheorem{conj}[thm]{Conjecture}
% \newtheorem{fact}[thm]{Fact}
% \newtheorem{hypo}[thm]{Hypothesis}
% \newtheorem{assum}[thm]{Assumption}
\newtheorem{prop}{Proposition}
% \newtheorem{crit}[thm]{Criterion}
% \theoremstyle{definition}
% \newtheorem{defn}[thm]{Definition}
% \newtheorem{exmp}[thm]{Example}
\newtheorem{rem}{Remark}
% \newtheorem{prin}[thm]{Principle}

\DeclareMathOperator{\Tr}{tr}
\newcommand\xdownarrow[1][2ex]{%
   \mathrel{\rotatebox{90}{$\xleftarrow{\rule{#1}{0pt}}$}}
}
\DeclareMathOperator{\End}{End}
\DeclareMathOperator{\Hom}{Hom}
\DeclareMathOperator{\id}{id}
\DeclareMathOperator{\vers}{vers}
\DeclareMathOperator{\trans}{Trans}
\DeclareMathOperator{\rot}{Rot}
\DeclareMathOperator{\rank}{rank}
\DeclareMathOperator{\sinc}{sinc}
\DeclareMathOperator{\Ad}{Ad}
\DeclareMathOperator{\ad}{ad}

%% The section below needs to be put at the end of this file to make citation links work with ieeeconf.cls
\makeatletter
\let\NAT@parse\undefined
\makeatother
\usepackage{hyperref}
\hypersetup{
    unicode=false,          % non-Latin characters in Acrobat’s bookmarks
    pdftoolbar=true,        % show Acrobat’s toolbar?
    pdfmenubar=true,        % show Acrobat’s menu?
    pdffitwindow=false,     % window fit to page when opened
    pdfstartview={FitH},    % fits the width of the page to the window
    pdftitle={John Nelson Chiasson Memorial Script},    % title
    pdfauthor={Aykut C. Satici}, % author
    % pdfsubject={Subject},   % subject of the document
    % pdfcreator={Creator},   % creator of the document
    % pdfproducer={Producer}, % producer of the document
    % pdfkeywords={keyword1, key2, key3}, % list of keywords
    pdfnewwindow=true,      % links in new PDF window
    colorlinks=true,       % false: boxed links; true: colored links
    linkcolor=magenta,          % color of internal links (change box color with linkbordercolor)
    linkbordercolor=orange,
    citecolor=blue,        % color of links to bibliography
    citebordercolor=green,
    filecolor=magenta,      % color of file links
    urlcolor=cyan,           % color of external links
    urlbordercolor=blue,
}

\usepackage{generic}

% Not sure if needed
\def\BibTeX{{\rm B\kern-.05em{\sc i\kern-.025em b}\kern-.08em
    T\kern-.1667em\lower.7ex\hbox{E}\kern-.125emX}}

\markboth{\journalname, VOL. XX, NO. XX, May 2025}
{Satici: John Chiasson Memorial Talk Script
(May 2025)}

\begin{document}

\title{Script: John Nelson Chiasson} 
\author{
    Aykut C. Satici \IEEEmembership{Member, IEEE}
%    \thanks{Not submitted for review in May 2025.}
    \thanks{A. C. Satici is with the Mechanical and Biomedical Engineering Department, Boise State University, Boise, ID 83706 USA
    (e-mail: aykutsatici@boisestate.edu).}
}
\maketitle
% \IEEEpeerreviewmaketitle

\section{Slide 1}
%
My short presentation cannot do justice to this great man, John Nelson Chiasson, who passed away on February 26. 2025.

The title is taken from a Facebook page that I stumbled upon approximately $4$ 
years ago - John Chiasson: An American Treasure. It was such a great description
that I printed out a few pictures from this FB page and hung them 
on the desk he used to work at with his mentee, Mansi, who is also with us
today, having travelled from California.

\section{Slide 2}

I got to know John when I interviewed here at Boise State back in Fall 2016. 
During my job talk, John noticed a mistake in one of my slides and pointed it out to me.

Much later, I got to meet some of the members of his family; Laura and Tim, in 
Istanbul. I met his twin brother Bill, about whom I had heard so much, when John
sadly passed away. Bill and Laura sent me a bunch of pictures from their stack, 
which I really appreciated and sprinkled throughout this presentation. This is 
the phase of John's life I had no idea about. Look how fit and handsome he was 
in his youth!

Bill mentioned how important this wand was to both him and John. They found this
wand during the time they spent in Africa while their father was taking a
sabbatical there.

Even though John and I met 8-9 years ago, we shared a lot of fond memories. This
picture on the left is taken at Mountain Home, where there is a US air base. We 
have been working on several projects in collaboration with Pitch Aeronautics, 
whose CEO, Zach Adams, is also an F-15 pilot! He had invited us for his last 
flight before he took a sabbatical year from the Air Force.

The picture on the right is from last year, when John invited my wife and me to 
his house and cooked us a dinner, whose recipe he learned from his Swiss friend,
Thomas Keller.

This picture on the left is taken in Stanley, ID, when we visited in 2021, with 
our research team. We were there just this last Saturday to disperse John's 
ashes at one of his favorite places to visit.

\section{Slide 3}

John was, among other things, a brilliant academician. The seeds of this brilliance were planted when he attended the University of Arizona to study 
mathematics. He used to tell me that he always liked math because everything was
well-defined and there was no room for hand-wavy arguments - one of John's most
important traits:

\begin{quote}
John was a rigorous man - he did not tolerate artificial, phony arguments or
opinions.
\end{quote}

He then switched gears to become a master engineer at Washington State
University, and a doctor of Control Sciences at Unviersity of Minnesote Twin
Cities.

\section{Slide 4}

John went on to produce a brilliant academic career. He was a top $1\%$ scientist according to \ldots had an h-index of $52$!

I had the honor and pleasure to produce his final journal paper, working with
him on the topic of state estimation and drone navigation around power lines
using the magnetic field produced by the three-phase currents. This was a
project we completed in collaboration with Pitch Aeronautics.

  
% \begin{abstract} % Abstract of not more than 200 words.
    This technical report demonstrates the implementation of a decision 
    transformer (DT) for learning the shortest paths on graphs.
\end{abstract}

% \begin{IEEEkeywords}
%     decision transfomer, machine learning, generative pretraining
% \end{IEEEkeywords}

% makeindex root.nlo -s nomencl.ist -o root.nls
% \nomenclature[aa]{$a, b$}{Indices referring to arbitrary frames (e.g. 
$\Sigma_A$, $\Sigma_B$)}
\nomenclature[ab]{$\Sigma_0$}{Inertial coordinate frame with origin $o_0$}
\nomenclature[ab]{$\Sigma_P$}{Pole coordinate frame with origin $o_p$}
\nomenclature[ac]{$\Sigma_B$}{Drone coordinate frame at the CoM $o_b$}
\nomenclature[ad]{$\Sigma_C$}{Contact frame with origin $o_c$}
\nomenclature[ae]{$\bm{p}_{ab}$}{Pos. vector from $o_a$ to $o_b$}
\nomenclature[af]{$\bm{R}_{ab}$}{Rotation matrix transforming ($\Sigma_B
\rightarrow \Sigma_A$)}
\nomenclature[ag]{$\bm{T}_{ab}$}{Hom. transformation ($\Sigma_B \rightarrow 
\Sigma_A$)}
\nomenclature[ah]{$\bm{v}_{ab}$}{Body linear velocity of $\Sigma_B$ 
($=\bm{R}_{ab}^\top \dot{\bm{p}}_{ab}$)}
\nomenclature[ai]{$\bm{\omega}_{ab}$}{Body angular velocity of $\Sigma_B$ w.r.t. 
$\Sigma_A$}
\nomenclature[aj]{$\bm{V}_{ab}$}{Body twist of $\Sigma_B$ w.r.t. 
$\Sigma_A$($= \bmat{\bm{\omega}^\top_{ab} & \bm{v}^\top_{ab}}^\top$)} 
\nomenclature[ba]{$\bm{f}$}{Drone thrust force ($ = f \bm{z}_b$, $f < 0$, 
specified in $\Sigma_B$)}
\nomenclature[bb]{$\bm{\tau}$}{Drone torque (specified in $\Sigma_B$)}
\nomenclature[bc]{$\bm{\mc{F}}$}{Drone wrench ($= \bmat{ 
\bm{\tau}^\top & \bm{f}^\top}^\top$)}
\nomenclature[ca]{$\bm{f}_e$}{Contact force on the drone (specified in 
$\Sigma_C$)}
\nomenclature[cb]{$\bm{\tau}_e$}{Contact torque on the drone (specified in 
$\Sigma_C$)}
\nomenclature[cc]{$\bm{\mc{F}}_e$}{Contact wrench on the drone (specified in 
$\Sigma_C$)}
\nomenclature[za]{$\bm{g}$}{Gravitational acceleration vector (specified in
$\Sigma_0$)}
\nomenclature[zb]{$m$}{Mass of the drone}
\nomenclature[zc]{$\bm{J}$}{Inertia matrix of the drone (specified in
$\Sigma_B$)}
\nomenclature[zd]{$m_w$}{Mass of the wire}
\nomenclature[ze]{$J_w$}{Inertia matrix of the wire about the pole along the
$\bm{y}_p-$axis}
\printnomenclature


\end{document}
